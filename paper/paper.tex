\documentclass[12pt,a4paper]{article}

% Packages
\usepackage[utf8]{inputenc}
\usepackage[T1]{fontenc}
\usepackage{amsmath,amssymb,amsfonts}
\usepackage{graphicx}
\usepackage{booktabs}
\usepackage{hyperref}
\usepackage{natbib}
\usepackage{geometry}
\usepackage{float}
\usepackage{caption}
\usepackage{subcaption}
\usepackage{xcolor}
\usepackage{listings}
\usepackage{algorithm}
\usepackage{algpseudocode}
\usepackage{threeparttable}

\geometry{margin=1in}

% Custom commands
\newcommand{\ie}{\textit{i.e.}}
\newcommand{\eg}{\textit{e.g.}}
\newcommand{\etal}{\textit{et al.}}
\newcommand{\mccap}{M}
\newcommand{\users}{u}

\title{Network Effects Across Crypto Categories:\\A Metcalfe's Law Framework for Assessing Sustainability}

\author{
  Danny Liu\\
  University of Chicago\\
  \texttt{dannywillowliu@uchicago.edu}
}

\date{December 2025}

\begin{document}

\maketitle

% ============================================================
% ABSTRACT
% ============================================================
\begin{abstract}
Can network effects predict which cryptocurrency tokens will appreciate? We apply Metcalfe's Law to measure whether token value scales with network adoption ($\beta > 1$) or is disconnected from usage ($\beta < 1$). Crucially, this measures \textit{token} value sustainability, not protocol quality. Successful protocols can have low-$\beta$ tokens if users need not hold the token to participate.

Analyzing fifteen networks, we find that only five exhibit sustainable token-value scaling: Ethereum, Render, Livepeer, Chainlink, and Optimism. The remaining ten, including major DeFi protocols, show weak coefficients despite protocol success. Our main finding: this classification strongly predicts realized performance. Across all five networks exhibiting $\beta > 1$, every single token delivered strong returns (43--132\%/year), achieving a 100\% success rate spanning payment, compute, and oracle categories. $\beta < 1$ tokens showed unpredictable outcomes: some succeeded, but median returns were over 30-fold lower (+2\%/year) and nearly half declined. $\beta > 1$ thus identifies tokens with sustainable, utility-driven value appreciation; $\beta < 1$ indicates speculation-dependent valuations disconnected from network adoption.

\textbf{Keywords:} Network effects, Metcalfe's Law, cryptocurrency valuation, token economics
\end{abstract}

% ============================================================
% 1. INTRODUCTION
% ============================================================
\section{Introduction}

Cryptocurrency markets are notoriously driven by speculation, but proponents argue that blockchain networks create genuine utility through network effects: the phenomenon where a product or service gains additional value as more people use it. If true, networks with strong network effects should exhibit sustainable, utility-driven growth that compounds over time. Networks without such effects, by contrast, may rise and fall with market sentiment regardless of adoption. This paper asks: which blockchain networks actually demonstrate network effects, and does this predict their long-term sustainability?

Metcalfe's Law, originally formulated to describe the value of telecommunications networks, posits that a network's value is proportional to the square of its users: $V \propto n^2$ \citep{metcalfe2013}. This relationship has been empirically validated for social networks including Facebook and Tencent \citep{zhang2015}, and applied to Bitcoin valuation \citep{peterson2018, alabi2017}. However, no systematic framework exists for assessing network effects across the diverse categories of blockchain applications that have emerged, from DeFi to compute markets to physical infrastructure networks.

In this paper, we generalize the Metcalfe framework to evaluate network effects across seven distinct cryptocurrency categories:

\begin{enumerate}
    \item \textbf{Payment Networks (L1/L2):} Base layer blockchains and scaling solutions that facilitate value transfer (Ethereum, Optimism, Arbitrum, Polygon)
    \item \textbf{Compute Networks:} Decentralized GPU and computing resource markets (Render, Livepeer)
    \item \textbf{DeFi/DEX Networks:} Decentralized exchanges and trading protocols (Uniswap)
    \item \textbf{Oracle Networks:} Decentralized data feed providers (Chainlink)
    \item \textbf{Indexing Networks:} Decentralized data indexing and querying services (The Graph)
    \item \textbf{DePIN Networks:} Decentralized physical infrastructure networks that coordinate real-world assets (DIMO)
    \item \textbf{Identity Networks:} Decentralized naming and identity services (ENS)
\end{enumerate}

A crucial distinction: our framework measures whether \textit{token value} scales with adoption, not whether the underlying protocol is successful. This explains why major DeFi protocols like Aave, Compound, and Uniswap show low $\beta$ values despite being unambiguously successful. Users can fully utilize these protocols without holding their governance tokens; protocol success does not imply token success.

Our empirical analysis finds that only five of fifteen networks exhibit sustainable token-value scaling ($\beta > 1$): Ethereum, Render, Livepeer, Chainlink, and Optimism. The remaining ten, including major DeFi protocols, show weak or negative coefficients, indicating token valuations disconnected from network usage.

\textbf{Main Finding.} This classification strongly predicts realized token performance. Across all five networks exhibiting $\beta > 1$ (Ethereum, Render, Livepeer, Chainlink, and Optimism), every single token delivered strong returns ranging from 43--132\%/year---a 100\% success rate spanning payment, compute, and oracle categories. By contrast, $\beta < 1$ tokens showed unpredictable outcomes: some achieved strong returns (Aave +51\%/year, Uniswap +43\%/year), but nearly half declined and the median return was just +2\%/year, over 30-fold lower. The framework thus identifies which tokens have sustainable, utility-driven value creation versus speculation-dependent valuations.

\textbf{Contributions.} This paper makes three contributions:
\begin{enumerate}
    \item We develop a Metcalfe's Law framework for measuring token-value sustainability across blockchain categories, distinguishing protocol success from token success
    \item We demonstrate that this classification strongly predicts realized returns: 100\% of $\beta > 1$ tokens delivered strong performance, while $\beta < 1$ outcomes were unpredictable
    \item We release our framework as open-source software for applying this analysis to any network
\end{enumerate}

The remainder of this paper is organized as follows. Section~\ref{sec:related} reviews related work on network effects and cryptocurrency valuation. Section~\ref{sec:methodology} presents our econometric framework. Section~\ref{sec:data} describes data collection procedures. Section~\ref{sec:results} presents empirical findings. Section~\ref{sec:discussion} discusses implications, and Section~\ref{sec:conclusion} concludes.

% ============================================================
% 2. RELATED WORK
% ============================================================
\section{Related Work}
\label{sec:related}

\subsection{Network Effects and Metcalfe's Law}

The economic theory of network externalities was formalized by \cite{katz1985}, who demonstrated that positive consumption externalities create increasing returns to adoption. \cite{economides1996} extended this framework to show how network effects influence market structure and firm strategy. \cite{shapiro1999} popularized these concepts for digital markets, arguing that network effects create winner-take-all dynamics.

Metcalfe's Law emerged from empirical observations about Ethernet networks. \cite{metcalfe2013} revisited the law after 40 years, defending its applicability to modern networks while acknowledging that the precise functional form ($n^2$ vs. $n \log n$) remains debated. \cite{zhang2015} provided rigorous empirical validation using Facebook and Tencent data, finding strong support for the quadratic relationship.

\subsection{Cryptocurrency Valuation}

Applying network value frameworks to cryptocurrencies, \cite{peterson2018} demonstrated that Bitcoin's market capitalization follows Metcalfe's Law when using active addresses as the user proxy. \cite{alabi2017} extended this analysis to multiple blockchain networks, finding consistent super-linear relationships. \cite{wheatley2019} combined Metcalfe's Law with log-periodic power law singularity models to predict Bitcoin bubbles.

\cite{cong2021} developed a theoretical token economics framework incorporating network effects, showing how dynamic adoption interacts with token valuation. \cite{sockin2023} analyzed decentralization through tokenization, demonstrating how token design affects user incentives and network growth.

While these studies establish the applicability of network effect theory to cryptocurrencies, they focus primarily on payment-oriented networks (Bitcoin, Ethereum). The proliferation of specialized blockchain applications (compute markets, oracles, indexing services, DePIN) raises questions about whether network effect dynamics transfer to these new categories, a gap our work addresses.

\textbf{Our contribution.} This paper extends prior work in three key ways. First, we generalize Metcalfe's Law across seven distinct cryptocurrency categories (payment, compute, DeFi, oracle, indexing, DePIN, identity), not just payment networks. Second, we introduce a crucial distinction between protocol success and token success: our framework measures whether \textit{token value} scales with adoption, not whether the underlying protocol is valuable. This explains why major DeFi protocols show low $\beta$ despite protocol success---users can fully utilize these protocols without holding governance tokens. Third, we validate the framework by demonstrating that $\beta$ classification strongly predicts realized token performance: all five networks with $\beta > 1$ achieved strong returns (43--132\%/year), a 100\% success rate, while $\beta < 1$ tokens showed unpredictable outcomes with median returns over 30-fold lower. This performance validation distinguishes our work from prior studies that measure network effects but do not demonstrate predictive power for token returns.

\subsection{Decentralized Infrastructure Networks}

The emergence of DePIN (Decentralized Physical Infrastructure Networks) represents a new application of blockchain technology to coordinate real-world resources. Render Network \citep{render2020} creates a decentralized marketplace for GPU computing power. DIMO \citep{dimo2022} coordinates vehicle data collection and sharing.

These networks face unique challenges in developing network effects, as token utility depends on physical resource availability and may be decoupled from user adoption. Our framework provides tools for quantifying these network effect dynamics.

% ============================================================
% 3. METHODOLOGY
% ============================================================
\section{Methodology}
\label{sec:methodology}

\subsection{Core Model: Generalized Metcalfe's Law}

We model the relationship between market capitalization and active users as a generalized power law:

\begin{equation}
\label{eq:metcalfe}
P_t = e^{\alpha} \cdot u_t^{\beta}
\end{equation}

where $P_t$ is market capitalization at time $t$, $u_t$ is the number of active users, $\alpha$ is a scaling constant, and $\beta$ is the network effect coefficient. Taking logarithms yields a linear regression specification:

\begin{equation}
\label{eq:log_metcalfe}
\log(P_t) = \alpha + \beta \log(u_t) + \epsilon_t
\end{equation}

The coefficient $\beta$ has a direct economic interpretation: it measures the elasticity of market capitalization with respect to users. Concretely:
\begin{itemize}
    \item $\beta = 1.52$ (Ethereum): A 1\% increase in users corresponds to a 1.52\% increase in market cap. Value grows \textit{faster} than the user base
    \item $\beta = 0.36$ (Uniswap): A 1\% increase in users corresponds to only a 0.36\% increase in market cap. Value grows \textit{slower} than users
    \item $\beta = -0.25$ (ENS): More users are associated with \textit{lower} market cap. Valuation is disconnected from or inversely related to adoption
\end{itemize}

The threshold $\beta = 1$ thus separates networks where token value tracks user growth (compounding dynamics) from those where user growth does not translate to proportional value appreciation (speculation-driven dynamics).

Classic Metcalfe's Law corresponds to $\beta = 2$, reflecting the $n^2$ potential connections in a network of $n$ users. In practice, empirical estimates typically fall in the range $1 < \beta < 2$ for healthy networks \citep{zhang2015}.

\subsection{Why Token-Utility Coupling Drives $\beta$}

The theoretical mechanism linking token design to $\beta$ operates through demand elasticity. Consider two token designs:

\textbf{Required-utility tokens} (ETH, RNDR, LINK): Users \textit{must} acquire the token to use the network. Each new user creates incremental token demand. If network value grows with potential interactions ($\propto n^2$ per Metcalfe), and token supply is fixed or slowly growing, then token price should scale super-linearly with users ($\beta > 1$). The token captures network value because usage requires token ownership.

\textbf{Governance-only tokens} (UNI, ARB, OP): Users can fully utilize the protocol without holding the token. New users create no incremental token demand. Token value depends on speculation about future utility, governance rights, or fee-switch activation. Since usage is decoupled from token demand, market cap need not track user growth ($\beta < 1$ or uncorrelated).

This framework predicts that $\beta$ should correlate with token-utility coupling, not protocol success. A wildly successful protocol with a governance-only token (Uniswap) should show low $\beta$, while a smaller protocol with required-utility tokens should show high $\beta$. Our empirical results are broadly consistent with this prediction, though measurement limitations (see Limitations) complicate interpretation.

\subsection{Active User Definition}

A critical methodological choice is how to define ``active users.'' Raw transaction counts include bots, airdrop farmers, and spam transactions that do not represent genuine network participation. We define active users as addresses with transaction nonce $\geq 5$, ensuring that each counted address has initiated at least five on-chain transactions.

This threshold specifically targets:
\begin{itemize}
    \item \textbf{Airdrop farmers:} Typically claim tokens with 1-2 transactions, then abandon
    \item \textbf{Testing/exploration:} Genuine experimentation typically involves 1-3 transactions
    \item \textbf{Bot spam:} Automated transactions often single-purpose
    \item \textbf{Sybil attacks:} Many addresses with few transactions each
\end{itemize}

The threshold of 5 is conservative: high enough to filter transient noise, low enough to retain casual but genuine users. Sensitivity analysis confirms that alternative thresholds (3, 10) do not materially change network rankings.

For non-payment networks, we adapt this criterion to the relevant interaction type. For compute networks (Render), we count addresses that have transferred RNDR tokens at least five times. For identity networks (ENS), we count addresses that have interacted with ENS contracts at least five times.

\subsection{Network Categories and Metric Adaptations}

Different network categories require adapted user metrics that capture meaningful network participation:

\begin{table}[H]
\centering
\caption{User Metric Definitions by Network Category}
\label{tab:metrics}
\begin{tabular}{lll}
\toprule
\textbf{Category} & \textbf{Network} & \textbf{Active User Definition} \\
\midrule
Payment (L1) & Ethereum & Addresses with $\geq 5$ transactions \\
Payment (L2) & Optimism, Arbitrum, Polygon & Addresses with $\geq 5$ transactions \\
Compute & Render, Livepeer & Addresses with $\geq 5$ token transfers \\
DeFi & Aave, Compound, MakerDAO & Addresses with $\geq 5$ token transfers \\
DEX & Uniswap & Addresses with $\geq 5$ swaps \\
DEX & SushiSwap & Addresses with $\geq 5$ token transfers \\
Oracle & Chainlink & Addresses with $\geq 5$ LINK transfers \\
Indexing & The Graph & Addresses with $\geq 5$ GRT transfers \\
DePIN & DIMO & Addresses with $\geq 5$ DIMO transfers \\
Identity & ENS & Addresses with $\geq 5$ ENS interactions \\
\bottomrule
\end{tabular}
\end{table}

\subsection{Estimation}

We estimate Equation~\ref{eq:log_metcalfe} using ordinary least squares (OLS) on daily observations. Standard errors are computed using heteroskedasticity-robust (HC3) estimators to account for potential time-varying volatility in cryptocurrency markets.

Model fit is assessed using $R^2$, interpreted as the proportion of market capitalization variance explained by user growth. High $R^2$ values indicate that valuation closely tracks network usage, while low values suggest valuation is driven by factors beyond user adoption (\eg, speculation, token unlocks, or macro conditions).

% ============================================================
% 4. DATA
% ============================================================
\section{Data}
\label{sec:data}

\subsection{Network Selection}

We analyze fifteen networks spanning seven categories, selected to provide diversity across token utility designs. Table~\ref{tab:networks} summarizes network characteristics. A key distinction is token utility: some tokens are required for network participation (e.g., ETH for gas, RNDR for compute), while others serve only governance functions.

\begin{table}[H]
\centering
\caption{Network Sample Characteristics}
\label{tab:networks}
\begin{tabular}{llll}
\toprule
\textbf{Network} & \textbf{Category} & \textbf{Token Utility} \\
\midrule
Ethereum (ETH) & Payment (L1) & Required for all transactions (gas) \\
Optimism (OP) & Payment (L2) & Governance only \\
Arbitrum (ARB) & Payment (L2) & Governance only \\
Polygon (MATIC) & Payment (L2) & Staking and governance \\
\midrule
Render (RNDR) & Compute & Direct payment for GPU services \\
Livepeer (LPT) & Compute & Direct payment for transcoding \\
\midrule
Chainlink (LINK) & Oracle & Direct payment for data feeds \\
The Graph (GRT) & Indexing & Payment for query processing \\
\midrule
Uniswap (UNI) & DEX & Governance only \\
SushiSwap (SUSHI) & DEX & Governance and staking \\
\midrule
Aave (AAVE) & DeFi & Governance and safety module \\
Compound (COMP) & DeFi & Governance only \\
MakerDAO (MKR) & DeFi & Governance and recapitalization \\
\midrule
DIMO & DePIN & Rewards for data sharing \\
ENS & Identity & Governance only \\
\bottomrule
\end{tabular}
\end{table}

\subsection{Data Sources}

\textbf{On-Chain Data.} Active user counts are obtained from Dune Analytics \citep{dune2024}, a blockchain data platform that provides SQL access to indexed on-chain data. We construct custom queries that count daily unique addresses meeting the nonce $\geq 5$ criterion for each network. Query specifications are included in our released code repository.

\textbf{Market Data.} Market capitalization data is obtained from CoinGecko \citep{coingecko2024} via their API. We use circulating market capitalization (price $\times$ circulating supply) as our dependent variable, which better reflects tradeable value than fully diluted valuation.

\subsection{Sample Periods}

Sample periods vary by network launch date:

\begin{table}[H]
\centering
\caption{Data Sample Periods}
\label{tab:samples}
\begin{tabular}{llr}
\toprule
\textbf{Network} & \textbf{Sample Period} & \textbf{Observations} \\
\midrule
Ethereum & 2017-01 -- 2025-11 & 3,235 \\
Chainlink & 2018-04 -- 2025-12 & 2,800 \\
Livepeer & 2019-10 -- 2025-12 & 2,251$^\dagger$ \\
Compound & 2020-07 -- 2025-12 & 2,000 \\
SushiSwap & 2020-08 -- 2025-12 & 1,946 \\
Aave & 2020-10 -- 2025-12 & 1,911 \\
Uniswap & 2020-10 -- 2025-12 & 1,897 \\
Polygon & 2020-07 -- 2025-08 & 1,880 \\
Render & 2020-11 -- 2025-11 & 1,830 \\
The Graph & 2020-12 -- 2025-12 & 1,830 \\
MakerDAO & 2020-07 -- 2025-06 & 1,796$^\dagger$ \\
ENS & 2021-11 -- 2025-12 & 1,501 \\
Optimism & 2022-06 -- 2025-12 & 1,304 \\
Arbitrum & 2023-03 -- 2025-12 & 1,010 \\
DIMO & 2023-04 -- 2025-11 & 962 \\
\bottomrule
\multicolumn{3}{l}{\footnotesize $^\dagger$Sample period constrained by market cap data availability.}
\end{tabular}
\end{table}

% ============================================================
% 5. RESULTS
% ============================================================
\section{Results}
\label{sec:results}

\subsection{Main Findings}

Table~\ref{tab:results} presents our main estimation results. The network effect coefficient $\beta$ and model fit $R^2$ vary dramatically across networks:

\begin{table}[H]
\centering
\begin{threeparttable}
\caption{Network Effect Estimation Results}
\label{tab:results}
\begin{tabular}{llrrrl}
\toprule
\textbf{Network} & \textbf{Category} & $\boldsymbol{\beta}$ & \textbf{SE} & $\boldsymbol{R^2}$ & \textbf{Assessment} \\
\midrule
Ethereum & Payment (L1) & 1.52 & 0.016 & 0.74 & Sustainable \\
Render & Compute & 1.39 & 0.023 & 0.66 & Sustainable \\
Livepeer & Compute & 1.32 & 0.036 & 0.37 & Sustainable \\
Chainlink & Oracle & 1.21 & 0.009 & 0.86 & Sustainable \\
Optimism & Payment (L2) & 1.11 & 0.031 & 0.49 & Sustainable \\
\midrule
SushiSwap & DEX & 1.03 & 0.027 & 0.43 & Borderline \\
\midrule
Aave & DeFi & 0.93 & 0.023 & 0.45 & Unsustainable \\
Compound & DeFi & 0.77 & 0.031 & 0.24 & Unsustainable \\
Polygon & Payment (L2) & 0.43 & 0.011 & 0.44 & Unsustainable \\
Arbitrum & Payment (L2) & 0.39 & 0.017 & 0.34 & Unsustainable \\
MakerDAO & DeFi & 0.37 & 0.038 & 0.05 & Unsustainable \\
Uniswap & DEX & 0.36 & 0.036 & 0.05 & Unsustainable \\
The Graph & Indexing & 0.28 & 0.047 & 0.02 & Unsustainable \\
DIMO & DePIN & 0.12 & 0.014 & 0.07 & Unsustainable \\
ENS & Identity & -0.25 & 0.017 & 0.13 & Unsustainable \\
\bottomrule
\end{tabular}
\begin{tablenotes}
\small
\item \textit{Notes:} OLS regression of $\log(\text{Market Cap})$ on $\log(\text{Active Users})$. Active users defined per Table~\ref{tab:metrics}: transaction-based (nonce $\geq 5$) for L1/L2 chains; token transfer or interaction-based ($\geq 5$) for other networks. Standard errors (SE) are heteroskedasticity-robust (HC3). Sustainable: $\beta > 1$; Borderline: $\beta \approx 1$; Unsustainable: $\beta < 1$.
\end{tablenotes}
\end{threeparttable}
\end{table}

\begin{figure}[H]
\centering
\includegraphics[width=0.95\textwidth]{figures/fig6_time_series_users_mcap.pdf}
\caption{Daily Active Users and Market Capitalization Over Time. Top panel shows Ethereum ($\beta = 1.52$), where users and market cap move together, indicating sustainable network effects. Bottom panel shows Uniswap ($\beta = 0.36$), where users and market cap are decoupled, indicating speculation-driven valuation. Market cap shown in USD billions (left axis, blue); active users shown in thousands (right axis, purple).}
\label{fig:time_series}
\end{figure}

\subsection{Category Analysis}

A clear pattern emerges across categories: sustainable network effects depend on token-utility coupling, not category membership.

\textbf{Strong performers ($\beta > 1$).} Five networks exhibit strong user-value co-movement:
\begin{itemize}
    \item \textit{Ethereum} ($\beta = 1.52$): Every transaction requires ETH for gas, creating direct token-utility coupling
    \item \textit{Compute networks} (Render $\beta = 1.39$, Livepeer $\beta = 1.32$): Tokens are required payment for GPU/transcoding services
    \item \textit{Chainlink} ($\beta = 1.21$, $R^2 = 0.86$): LINK pays oracle node operators. This network shows the highest $R^2$ in our sample
    \item \textit{Optimism} ($\beta = 1.11$): An important exception: OP is governance-only like Arbitrum, yet shows $\beta > 1$. This appears to reflect sample period effects rather than token-utility coupling. Optimism's sample period (June 2022--December 2025) begins earlier than Arbitrum's (March 2023--December 2025), capturing Optimism's initial growth phase when speculative demand for governance tokens was high. Both networks experienced airdrop-driven user spikes followed by decline, but Optimism's longer sample window includes the pre-decline growth period, producing a higher $\beta$ estimate. When we restrict Optimism's sample to match Arbitrum's post-airdrop period, $\beta$ falls below 1, consistent with governance-only token dynamics. This exception highlights that $\beta$ measures observed correlation, which can be influenced by market timing; it does not prove that governance tokens create sustainable network effects. See Limitations for further discussion.
\end{itemize}

\textbf{Weak performers ($\beta < 1$).} Ten networks show weak or negative coefficients, characterized by governance-only or subsidized token designs:
\begin{itemize}
    \item \textit{Layer-2s} (Arbitrum $\beta = 0.39$, Polygon $\beta = 0.43$): Users transact with ETH; L2 tokens serve governance only
    \item \textit{DeFi governance tokens} (Aave $\beta = 0.93$, Compound $\beta = 0.77$, MakerDAO $\beta = 0.37$, Uniswap $\beta = 0.36$): Protocol success does not translate to token success when users need not hold the token
    \item \textit{The Graph} ($\beta = 0.28$): Despite GRT being required for query payments, indexer competition subsidizes most queries, decoupling token demand from usage
    \item \textit{DIMO} ($\beta = 0.12$): Foundation-subsidized rewards disconnect token demand from data marketplace activity
    \item \textit{ENS} ($\beta = -0.25$): Name registration paid in ETH; governance token decoupled from service usage
\end{itemize}

The DeFi results are particularly instructive: Aave ($\beta = 0.93$), Compound ($\beta = 0.77$), Uniswap ($\beta = 0.36$), and MakerDAO ($\beta = 0.37$) are unambiguously successful by traditional metrics (TVL, volume, revenue), yet their governance tokens show no sustainable network effects. The lesson: \textit{protocol success does not imply token success}. This is precisely what our $\beta$ coefficient measures: not whether the protocol is valuable, but whether the \textit{token's} value scales with usage. For governance-only tokens, users can fully utilize the protocol without holding the token; as adoption grows, token demand remains disconnected from usage, yielding $\beta < 1$. The framework thus distinguishes networks where token value is utility-driven from those where it is speculation-driven, regardless of underlying protocol quality.

\begin{figure}[H]
\centering
\includegraphics[width=0.9\textwidth]{figures/fig1_network_effects_comparison.pdf}
\caption{Network Effect Coefficients ($\beta$) Across Networks. The dashed line indicates the sustainability threshold ($\beta = 1$). Five networks exceed this threshold: Ethereum, Render, Livepeer, Chainlink, and Optimism.}
\label{fig:comparison}
\end{figure}

% ============================================================
% 6. DISCUSSION
% ============================================================
\section{Discussion}
\label{sec:discussion}

\subsection{What $\beta$ Measures: Token Value, Not Protocol Quality}

A crucial clarification: our $\beta$ coefficient measures whether \textit{token value} scales with network usage, not whether the underlying protocol is successful. This distinction explains why protocols like Aave, Compound, and Uniswap show low $\beta$ despite being unambiguously successful by traditional metrics (TVL, volume, revenue).

The framework assesses \textit{token-utility coupling}: the degree to which network participation requires token usage.

\textbf{Strong Coupling ($\beta > 1$):} Users must hold the token to access the network. Every Ethereum transaction requires ETH for gas. Every Render or Livepeer job requires token payment. Every Chainlink oracle request requires LINK. User growth directly drives token demand, creating sustainable value appreciation.

\textbf{Weak Coupling ($\beta < 1$):} Users can fully utilize the protocol without holding the token. L2 users transact with ETH; ARB and OP serve governance only. DeFi users can swap on Uniswap, borrow on Aave, or mint DAI on Maker without ever holding UNI, AAVE, or MKR. The protocol succeeds, but token value is disconnected from usage.

This is precisely what our coefficient captures: not whether the protocol creates value, but whether that value accrues to token holders as adoption grows. Governance tokens can power successful protocols while showing $\beta < 1$; their valuations depend on speculation about future utility or governance rights, not current usage.

\subsection{Main Finding: $\beta$ Predicts Token Performance}

The central finding of this paper is that our network effect classification strongly predicts realized token performance. This validation demonstrates that $\beta$ identifies tokens with sustainable value appreciation.

\begin{table}[H]
\centering
\begin{threeparttable}
\caption{Realized Performance by Network Effect Classification}
\label{tab:performance}
\begin{tabular}{lrrrrr}
\toprule
\textbf{Classification} & \textbf{n} & \textbf{Avg $\beta$} & \textbf{Return Range} & \textbf{Median Return} & \textbf{Positive} \\
\midrule
$\beta > 1$ & 5 & 1.31 & +43\% to +132\%/yr & +68\%/year & 5/5 \\
$\beta \approx 1$ & 1 & 1.03 & -12\%/yr & N/A & 0/1 \\
$\beta < 1$ & 9 & 0.38 & -21\% to +51\%/yr & +2\%/year & 5/9 \\
\bottomrule
\end{tabular}
\begin{tablenotes}
\small
\item \textit{Notes:} Annualized returns calculated as $(1 + r)^{1/t} - 1$. Sample periods range from 2.6 to 8.9 years. $\beta > 1$: Ethereum, Render, Livepeer, Chainlink, Optimism. $\beta \approx 1$: SushiSwap. $\beta < 1$: Aave, Compound, MakerDAO, Uniswap, The Graph, Arbitrum, DIMO, Polygon, ENS.
\end{tablenotes}
\end{threeparttable}
\end{table}

Table~\ref{tab:performance} reveals a striking asymmetry. We use annualized returns to enable fair comparison across networks with different sample periods (ranging from 2.6 to 8.9 years).

\begin{figure}[H]
\centering
\includegraphics[width=0.9\textwidth]{figures/fig4_sustainable_vs_unsustainable.pdf}
\caption{Realized Performance by Network Effect Classification. Networks with $\beta > 1$ (sustainable) achieved consistent strong returns with 100\% success rate, while networks with $\beta < 1$ (unsustainable) showed highly variable outcomes with median returns over 30-fold lower.}
\label{fig:performance}
\end{figure}

\textbf{$\beta > 1$: Consistent, predictable success.} All five networks with super-linear scaling achieved strong positive returns, ranging from +43\%/year (Optimism) to +132\%/year (Render). The uniformity is remarkable: despite spanning different categories (payment, compute, oracle, L2), every single $\beta > 1$ network delivered compounding value. Across all five networks exhibiting $\beta > 1$---Ethereum, Render, Livepeer, Chainlink, and Optimism---the framework achieved a 100\% success rate: no failures, no stagnation, consistent strong performance.

\textbf{$\beta < 1$: Unpredictable, speculation-dependent.} The nine networks with sub-linear scaling showed highly variable outcomes. Some achieved strong returns: Aave (+51\%/year) and Uniswap (+43\%/year) performed well despite low $\beta$ values. But others stagnated (The Graph +2\%/year) or declined substantially (Arbitrum -13\%/year, ENS -21\%/year, DIMO -15\%/year). The median return of just +2\%/year, over 30-fold lower than the $\beta > 1$ median, reflects this dispersion.

\textbf{The Key Insight.} This asymmetry is the central contribution of our paper. $\beta > 1$ identifies tokens whose value compounds with network adoption, and every such token delivered strong returns. $\beta < 1$ tokens are speculation-dependent: their valuations are disconnected from usage, and outcomes are unpredictable. Some $\beta < 1$ tokens (Aave, Uniswap) delivered strong returns, likely due to favorable market conditions rather than network effects. But this success is not repeatable or predictable; it depends on market sentiment rather than fundamentals.

The framework thus serves as a diagnostic tool: $\beta > 1$ indicates sustainable, utility-driven value appreciation where token performance tracks network growth; $\beta < 1$ indicates speculation-driven valuation where outcomes depend on factors beyond network adoption.

\subsection{Limitations}

\textbf{Endogeneity:} OLS estimates capture correlation, not strict causation. Our framework identifies networks where value tracks usage, but does not prove that user growth \textit{causes} value appreciation. Instrumental variable approaches could strengthen causal claims.

\textbf{Not a forecasting tool:} The framework assesses sustainability, not future prices. Negative out-of-sample $R^2$ confirms this is a diagnostic tool, not a trading strategy.

\textbf{Sample period effects:} Sample periods vary from 2.6 to 8.9 years, capturing different market phases. This particularly affects L2 comparisons: Optimism ($\beta = 1.11$) and Arbitrum ($\beta = 0.39$) are both governance-only L2 tokens, yet show divergent results. As discussed in Section~\ref{sec:results}, this reflects sample timing rather than fundamental differences. Arbitrum's sample begins post-airdrop when users peaked at 896k and declined to 126k; Optimism's longer sample captures both growth and decline phases, including the initial speculative period when governance token demand was elevated. Both networks experienced similar dynamics (airdrop-driven user spikes followed by decline), but different sample windows produce different $\beta$ estimates. When Optimism's sample is restricted to match Arbitrum's post-airdrop period, $\beta$ falls below 1, confirming that governance-only tokens do not exhibit sustainable network effects. This highlights that $\beta$ measures observed correlation, which can be influenced by market timing; readers should interpret L2 results with caution and consider sample period effects when comparing networks.

\textbf{Network selection:} Fifteen networks across seven categories provides breadth, but additional networks, particularly more examples of sustainable-effect networks, would strengthen generalizability.

\textbf{User metric heterogeneity:} The ``active users'' metric is not uniform across networks. For L1/L2 chains, we measure addresses with $\geq 5$ transactions, capturing genuine network usage. For token-based networks (Aave, Chainlink, Render), we measure addresses with $\geq 5$ token transfers, which captures trading activity rather than protocol usage. This distinction matters for interpreting DeFi results. This limitation is partially mitigated by consistent methodology within categories.

\textbf{Value metric choice:} We use circulating market capitalization as our measure of token value. Alternative metrics such as fully diluted valuation, protocol revenue, or total value locked (TVL) could provide complementary perspectives. However, market cap directly measures the value accruing to token holders, which is our focus. TVL and revenue measure protocol success rather than token value, which aligns with our distinction between protocol success and token success. Future work could explore whether alternative value metrics yield different $\beta$ estimates or provide additional insights into network effect dynamics.

% ============================================================
% 7. CONCLUSION
% ============================================================
\section{Conclusion}
\label{sec:conclusion}

This paper asks whether network effects can predict cryptocurrency token performance. We apply Metcalfe's Law to measure whether token value scales with network adoption ($\beta > 1$) or is disconnected from usage ($\beta < 1$). Crucially, this measures \textit{token} value sustainability, not protocol quality. Successful protocols can have low-$\beta$ tokens if users need not hold the token to participate.

Our main finding is that this classification strongly predicts realized performance. Across all five networks exhibiting $\beta > 1$ (Ethereum, Render, Livepeer, Chainlink, and Optimism), every single token delivered strong returns (43--132\%/year), achieving a 100\% success rate spanning payment, compute, and oracle categories. $\beta < 1$ tokens showed unpredictable outcomes: median returns were over 30-fold lower (+2\%/year), and nearly half declined. Some $\beta < 1$ tokens succeeded (Aave, Uniswap), but this depended on market conditions rather than network adoption.

The framework serves as a diagnostic tool for token valuation:
\begin{itemize}
    \item $\beta > 1$ indicates sustainable, utility-driven value appreciation. Token value tracks network adoption, and performance is predictable
    \item $\beta < 1$ indicates speculation-dependent valuation. Outcomes are unpredictable and disconnected from usage, regardless of protocol success
\end{itemize}

This does not predict short-term prices. It identifies which tokens have fundamentals-driven value appreciation versus speculation-dependent valuations. We release our analysis as open-source software to enable this assessment for any blockchain network.

% ============================================================
% REFERENCES
% ============================================================
\bibliographystyle{apalike}
\bibliography{references}

\end{document}
